\documentclass{article}

\usepackage{amssymb}

\renewcommand{\tt}{\ttfamily}
\newcommand{\codefont}{\small\tt}
\newcommand{\code}[1]{\mbox{\codefont{#1}}}
\newcommand{\ccode}[1]{``\code{#1}''}
\newcommand{\fcode}[1]{\mbox{\codefont{\footnotesize{#1}}}} % code in footnote

\usepackage{listings}
\lstset{aboveskip=0.7ex,
        belowskip=0.7ex,
        showstringspaces=false, % no special string space
        mathescape=true,
        xleftmargin=2ex,
        flexiblecolumns=false,
        basewidth=0.55em,
        basicstyle=\small\ttfamily}
\lstnewenvironment{curry}{\lstset{literate={->}{{$\rightarrow{}\!\!\!$}}3}}{}
\newcommand{\listline}{\vrule width0pt depth1.5ex}

\begin{document}

\begin{abstract}
  Needs a rather precise statement and result before writing the rest
  of the paper.
  \sloppy
\end{abstract}

\section{introduction}

dependent types

functional logic programming

dependent types in agda

goal: dependent types with ?

\section{semantics}
questions to answer:
  when do we evaluate types
  when do we backtrack
    in four of a kind example we need to check that refl doesn't work and then backtrack

    valid
------------------
set_i : set_{i+1}

A : set    x:A |- B : set
-------------------------
    (x:A) * B : set


A : set    x:A |- B : set
-------------------------
    (x:A) -> B : set

x : A \in \Gamma
------------------
  \Gamma |- x : A

s : A    t : B[x := s]
----------------------
  (s,t) : (x:A) * B


  t : (x:A) * B
----------------
     p1 t : A

     t : (x:A) * B
----------------------
  p2 t : B[x := p1 t]

       x : A |- t : B
--------------------------
   \ x . t : (x:A) -> B

  s : (x : A) -> B      t : A
--------------------------------
          s t : B[x := t]

  x : A       y : B
-----------------------
     x ? y : A ? B

    t : A ? B
----------------
 t : A    t : B

\section{examples}
\subsection{four of a kind}
\begin{verbatim}
  fourOfAKind [a1,a2,a3,a4,a5] = same [a1,a2,a3,a4] ? 
                                 same [a1,a2,a3,a5] ? 
                                 same [a1,a2,a4,a5] ? 
                                 same [a1,a3,a4,a5] ? 
                                 same [a2,a3,a4,a5]
 where same [x1,x2,x3,x4] = [x1,x2,x3,x4] == [x1,x1,x1,x1]

f : fourOfAKind [1,1,3,1,1] == tt
f = refl
\end{verbatim}

\subsection{sorting}
  algorithm
  introducing rewrite

\subsection{image editing}
  RGB vs Grayscale (two different types)
  properties
\begin{verbatim}
    f : (a ? b) -> x and g : a then in f g, f : a -> x
    f : x -> (a ? b) and g : a -> x then g . f : x -> x
\end{verbatim}
  downside
\begin{verbatim}
    f : (a ? b) -> (a ? b) then we expect if x : a then f x : a.
\end{verbatim}
This isn't true (could be solved with free variables and constraints).
What about
\begin{verbatim}
    f : (a -> a) ? (b -> b)
    f : t -> t where t = a ? b
\end{verbatim}

  


\section{conclusion}

\end{document}
